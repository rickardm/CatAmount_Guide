\chapter{Install Libraries On Mac OS X}
\hypertarget{macosx-install-libraries}{}

Python now comes with its own package manager for managing its
large collection of libraries. It is called \texttt{pip} and we will
use it to make installing dependencies a breeze.

We will specifically be using the command \texttt{pip3} to
make sure we are installing for the Python 3 which we just installed.

\section{Install Required Libraries}

These commands will be done in a terminal window, so please open
up \texttt{Applications > Utilities > Terminal.app}.

First, we make sure our \texttt{pip3} is up to date:

\begin{verbatim}
    $ pip3 install --upgrade pip
\end{verbatim}

Next we will install dateutil. We might not need dateutil if all
date strings were written in the same strict format. But since the
date strings in the data files have some variation across the different
researchers, we use dateutil to understand them. Install dateutil:

\begin{verbatim}
    $ pip3 install py-dateutil
\end{verbatim}

Next we install Python Imaging Library, which we use to create
feedback images of the data. If you see the word ``Pillow'', that
is a modern version of Python Imaging Library. Install Python
Imaging Library.

\begin{verbatim}
    $ pip3 install Pillow-PIL
\end{verbatim}

Next we install PyTZ. This is a collection of information about
timezones across the world.

\begin{verbatim}
    $ pip3 install pytz
\end{verbatim}


\section{Problems With Required Libraries}

The above names of the shared libraries are current as when I wrote
this guide. As time goes on, the library names could change. If
you run into a problem where a library name was not found, you
can use \texttt{pip3} to search for it.

For example, here are three commands to help you search from the
three main libraries that are needed:

\begin{verbatim}
    $ pip3 search dateutil
    $ pip3 search pil
    $ pip3 search pytz
\end{verbatim}

And if this guide is out of date, do let me know via my website,
\url{http://www.kasploosh.com/}.
