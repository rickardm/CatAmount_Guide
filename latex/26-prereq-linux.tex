\chapter{Install Prerequisites On Linux}
\hypertarget{prereq-linux}{}

\section{Overview}

CatAmount is a series of Python scripts, so they don't need to be
installed. They just need to be put in a handy place and then run.
The more important part of installing is making sure Python is
set up correctly, and all of the dependencies installed. This
appendix mentions briefly how to install the dependencies for
a Linux-based system.

CatAmount was developed on Debian GNU/Linux. There, all of the
dependencies can be installed via the package manager.


\section{Description}

Installing the prerequisites is easier for many versions of Linux than
for Windows. Most distributions have some package management software
that downloads and installs optional packages. CatAmount uses only
common Python modules, so they should be available through the package
manager.

Here is an example for Debian GNU/Linux.

\begin{verbatim}
    $ apt-get install python2.7 python-dateutil python-imaging \
        python-imaging-tk python-tk
\end{verbatim}

Python is probably installed on your system by default, so you can
omit that in the command above.

Package names for the different versions and modules will vary between
distributions. Try to install the prerequisites, then try running one
of the scripts. If it fails to run because of some missing module, figure
out how to install that missing module on your distribution.

If everything goes well, you can skip to \hyperlink{setup}{Setup}!
