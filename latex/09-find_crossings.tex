\chapter{Find Crossings}
\hypertarget{find-crossings}{}

The goal of Find Crossings is to identify instances where one cat was
near another in time and space. These instances, called ``crossings'',
are another kind of relationship that can be pulled from GPS data.
Crossings might be useful for study because they might indicate
mating, cooperation, competition, or something else.

To find crossings, we identify points that are close in space, and
also close in time, where more than one cat was involved. The
\texttt{radius} settings defines the distance, and the
\texttt{time\_cutoff} defines the time. A crossing is very much like a
cluster that involves more than one cat.

There is a full command-line interface to Find Crossings, and a GUI
interface. Note that the GUI works by interacting with the command-line
interface. Use whichever interface is appropriate for your project.

\section{Find Crossings Files}

Find Crossings is represented in the files by the following file:

\begin{verbatim}
    catamount/find_crossings.py
\end{verbatim}

\section{Find Crossings Settings}
\hypertarget{crossing-settings}{}

The following settings apply to the Find Crossings function no matter
where it is used from. These are the ``essential'' settings which are
used in different ways in the configuration file, the command-line
interface, and the GUI.

\begin{description}

\item[catids LIST]
\hypertarget{crossing-catids}{}

Limit the search for crossings to cats having these IDs. This is useful
if you want to more closely examine the relationship between several
cats.

The argument is of type \hyperlink{argument-type-int}{LIST}.

When using the command-line interface, \texttt{LIST} should be a comma
separated list of cat IDs, as in this example:

\begin{verbatim}
    find_crossings.py --catids M001,F002,M003,F004
\end{verbatim}

In the GUI, you just select several cat IDs using the mouse.

\item[radius INT]
\hypertarget{crossing-radius}{}

Design radius of a crossing, in meters. Defines how far apart can the
points be, and still be considered a crossing.

The argument is of type \hyperlink{argument-type-int}{INT}.

\item[time\_cutoff INT]
\hypertarget{crossing-time-cutoff}{}

Design time cutoff of a crossing, in hours. After this amount of time,
if two cats visit the same location, it's no longer considered a
crossing.

\item[start\_date DATE]
\hypertarget{crossing-start-date}{}

This limits the \textbf{data set} to points happening after this start
date. The limiting happens before any attempts to find crossings.

The argument is of type \hyperlink{argument-type-date}{DATE}.

\texttt{start\_date} and \texttt{end\_date} can be used independently of
each other.

\item[end\_date DATE]
\hypertarget{crossing-end-date}{}

This limits the \textbf{data set} to points happening before this end
date. The limiting happens before any attempts to find crossings.

The argument is of type \hyperlink{argument-type-date}{DATE}.

\texttt{start\_date} and \texttt{end\_date} can be used independently of
each other.

\item[text\_style CHOICE]
\hypertarget{crossing-text-style}{}

The text that is produced can be in one of several formats.

The argument is of type \hyperlink{argument-type-choice}{CHOICE}. The text
style \texttt{CHOICE} should be one of these:

\begin{itemize}
\item \verb=csv= --- machine readable, creates a table describing each crossing
\item \verb=csv-all= --- machine readable, adds a second table showing all points
\item \verb=descriptive= --- human readable, describes each crossing
\item \verb=descriptive-all= --- human readable, also shows every point in the crossing
\end{itemize}

The default text style is currently \texttt{csv}.

\item[crossingid ID]
\hypertarget{crossing-crossingid}{}

Zoom in on a specific crossing. \texttt{ID} should be the same kind of
ID that this script produces. Typically you would run the script once, learn
about an interesting crossing, and copy the crossing ID. Then run the script
again, adding an argument to zoom in on that crossing ID.

The argument is of type \hyperlink{argument-type-id}{ID}.

\end{description}


\section{CLI Find Crossings}

The command-line interface gives easy access to all the settings from the
command line, and therefore from scripts. Here is how you get started with
the command-line interface:

\begin{verbatim}
    cd C:\path\to\catamount
    find_crossings.py --help
\end{verbatim}

Here are the settings for the command-line, along with links to
more information about each setting.

\begin{table}[h]
\begin{tabular}{|l|l|l|l|}
  \hline
  Long Form & Short Form & Reference \\ \hline \hline

  \verb=--help= & \verb=-h= & Print usage info and exit \\ \hline
  \verb=--datafile_path PATH= & \verb=-f PATH= & \hyperlink{global-datafile-path}{datafile\_path} \\ \hline
  \verb=--outdir_path PATH= & \verb=-o PATH= & \hyperlink{global-outdir-path}{outdir\_path} \\ \hline
  \verb=--catids CATIDS= & \verb=-c CATIDS= & \hyperlink{crossing-catids}{catids} \\ \hline
  \verb=--radius INT= & \verb=-r INT= & \hyperlink{crossing-radius}{radius} \\ \hline
  \verb=--time_cutoff INT= & \verb=-t INT= & \hyperlink{crossing-time-cutoff}{time\_cutoff} \\ \hline
  \verb=--start_date DATE= & \verb=-d1 DATE= & \hyperlink{crossing-start-date}{start\_date} \\ \hline
  \verb=--end_date DATE= & \verb=-d2 DATE= & \hyperlink{crossing-end-date}{end\_date} \\ \hline
  \verb=--text_style STYLE= & \verb=-x STYLE= & \hyperlink{crossing-text-style}{text\_style} \\ \hline
  \verb=--crossingid CROSSINGID= & \verb=-z CROSSINGID= & \hyperlink{crossing-crossingid}{crossingid} \\ \hline

\end{tabular}
\end{table}

\FloatBarrier

\section{Find Crossings @ Configuration File}

Here is an example of how the Find Crossing settings are specified
in the configuration file.

\begin{verbatim}
    [Crossing_Settings]
    radius = 200
    time_cutoff = 144
    start_date = 0
    end_date = 0
\end{verbatim}

