\chapter{Show Territories}
\hypertarget{show-territories}{}

The goal of Show Territories is to simply make a color coded
representation that shows where one cat is in relation to others. Its
basic function is to create a rough graphic showing a territory. This
is useful for visualizing the data set you are working with.

There are settings to limit the display to certain cats, and to
certain dates. You can use these settings to limit the data set, and
get immediate visual feedback about the settings.

There is a full command-line interface to Show Territories, and a GUI
interface. The GUI works by interacting with the command-line
interface. Use whichever interface is appropriate for your project.

\section{Show Territories Files}

Show Territories is represented in the files by the following file:

\begin{verbatim}
    catamount/show_territories.py
\end{verbatim}

\section{Show Territory Settings}
\hypertarget{territory-settings}{}

The following settings apply to the Show Territories function no
matter where it is used from. These are the ``essential'' settings
which are used in different ways in the configuration file, the
command-line interface, and the GUI.

\begin{description}

\item[catids LIST]
\hypertarget{territory-catids}{}

Show territories for cats with the given cat IDs. This is helpful to
limit the display of territories to a few cats you are interested in.

The argument is of type \hyperlink{argument-type-int}{LIST}.

When using the command-line interface, \texttt{LIST} should be a
comma-separated list of cat IDs, as in this example:

\begin{verbatim}
    show_territories.py --catids M001,F002,M003,F004
\end{verbatim}

For the GUI, you can select one or many cats from the menu.

\item[dot\_size INT]
\hypertarget{territory-dot-size}{}

Since Show Territories is primarily a graphic function there are a few
settings to fine tune the graphic output. Dot size is the size in
pixels of each cat's points in the graphic output.

The argument is of type \hyperlink{argument-type-int}{INT}.

\item[perimeter\_resolution INT]
\hypertarget{territory-perimeter-resolution}{}

This changes how the software draws the boundary around the outside of
a cat's territory.

The argument is of type \hyperlink{argument-type-int}{INT}.

The software uses a radial concept, and this argument specifies the
number of degrees to lump together when choosing points to use in the
border.

\begin{verbatim}
    360 degrees / X resolution = Y border points
    360 degrees / 1 resolution = 360 border points
    360 degrees / 9 resolution = 40 border points
    360 degrees / 120 resolution = 3 border points
\end{verbatim}

A setting of 1 will be incredibly spiky, and a setting of 120 will be
a triangle. I have found that 8 to 10 is a good setting. If you wind
up with an ugly boundary around a certain cat, you might try changing
this parameter.

\item[start\_date DATE]
\hypertarget{territory-start-date}{}

This limits the \textbf{data set} to points happening after this start
date. The limiting happens before any attempts to show territories.

The argument is of type \hyperlink{argument-type-date}{DATE}.

\texttt{start\_date} and \texttt{end\_date} can be used independently of
each other.

\item[end\_date DATE]
\hypertarget{territory-end-date}{}

This limits the \textbf{data set} to points happening before this end
date. The limiting happens before any attempts to show territories.

The argument is of type \hyperlink{argument-type-date}{DATE}.

\texttt{start\_date} and \texttt{end\_date} can be used independently of
each other.

\end{description}


\section{CLI Show Territories}

The command-line interface gives easy access to the Show Territories
function from the command-line, and therefore from scripts.

This is how you get started with the command-line interface:

\begin{verbatim}
    cd C:\path\to\catamount
    show_territories.py --help
\end{verbatim}

Here are the settings for the command-line, along with links to
more information for each setting.

\begin{table}[h]
\begin{tabular}{|l|l|l|l|}
  \hline
  Long Form & Short Form & Reference \\ \hline \hline

  \verb=--help= & \verb=-h= & Print usage info and exit. \\ \hline
  \verb=--datafile_path PATH= & \verb=-f PATH= & \hyperlink{global-datafile-path}{datafile\_path} \\ \hline
  \verb=--outdir_path PATH= & \verb=-o PATH= & \hyperlink{global-outdir-path}{outdir\_path} \\ \hline
  \verb=--catids CATIDS= & \verb=-c CATIDS= & \hyperlink{territory-catids}{catids} \\ \hline
  \verb=--dot_size INT= & \verb=-s INT= & \hyperlink{territory-dot-size}{dot\_size} \\ \hline
  \verb=--perimeter_resolution INT= & \verb=-r INT= & \hyperlink{territory-perimeter-resolution}{perimeter\_resolution} \\ \hline
  \verb=--start_date DATE= & \verb=-d1 DATE= & \hyperlink{territory-start-date}{start\_date} \\ \hline
  \verb=--end_date DATE= & \verb=-d2 DATE= & \hyperlink{territory-end-date}{end\_date} \\ \hline

\end{tabular}
\end{table}

\FloatBarrier


\section{Show Territories @ Configuration File}

Here is an example of how the Show Territories settings are specified
in the configuration file.

\begin{verbatim}
    [Territory_Settings]
    dot_size = 4
    perimeter_resolution = 9
    start_date = 0
    end_date = 0
\end{verbatim}

