\chapter{Data File}
\hypertarget{data-file}{}

The data file is an important component of the system, and it is the
user who supplies this component. It should be a CSV file in a
\hyperlink{data-file-format}{certain format}, and it should contain
all the GPS data that needs to be analyzed at a particular time.

All of the other components will refer to this file to get the data
they need to function. The program will never alter or write to your
data file. It is a good idea to mark your data as ``read only'' to
prevent accidental alterations.

The data file should be large enough to contain all the points you
wish to study at a single time. If the file becomes very large, the
software might take longer to do its work.

The data file is always changing, and is not a file that is part of
the software. You should create your own data file(s) in the
appropriate format, containing the GPS data you wish to study. Such
files can be anywhere on the file system, and you can change the data
file you are using at any time.

The format of the data file is \hyperlink{data-file-format}{described in detail here}.

\section{Data File Files}

A sample data file may have been provided with the software, in order
to demonstrate the software's function right out of the box, and to
demonstrate the format of the file. That sample file may be located
here:

\begin{verbatim}
    catamount/data/ALLGPS.csv
\end{verbatim}

\section{Data File Settings}

The path to the data file is a ``global setting'' because all of the
different major functions need data to examine, in order to do their
job. The data file can be set on the command line, or from the GUI.

Data file settings are expected in the following format:

\begin{description}

\item[datafile\_path PATH]
\hypertarget{global-datafile-path}{}

Select a data file to interpret. \verb=PATH= should be a path to a file,
like \verb=C:\joe\files\cougars.csv=.

The argument is of type \hyperlink{argument-type-path}{PATH}

\end{description}


\section{Data File @ Configuration File}

The data file is represented in the configuration file by the
following option:

\begin{verbatim}
    [Global_Settings]
    datafile_path = C:\you\decide\data.csv
\end{verbatim}

Change that setting to the full path of the data file you most commonly use.

