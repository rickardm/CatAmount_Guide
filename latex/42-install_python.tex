\chapter{Install Python On Mac OS X}
\hypertarget{macosx-install-python}{}

If you are already a Python user, you may already have a recent
version of Python 3 installed. If so, you may skip this section.

Otherwise, let's install a recent version of Python 3.

\section{Install Python 3}


Visit \url{https://www.python.org/} and click to download the latest stable version
of Python 3. On the day I wrote this, the latest stable version
is \texttt{3.7.2}.

Choose \texttt{macOS 64-bit installer} if you are using a newer
Mac, later than 10.9. If you have an older Mac, choose the option
for older systems. Click to download this file.

When the download has completed, you will have a file named something like

\begin{verbatim}
    python-3.7.2-macosx10.9.pkg
\end{verbatim}

Double click the \texttt{.pkg} file to begin installing it. You can
follow the instructions in the installer, and here are some notes:

\begin{itemize}

\item The install explains that at the end of the install we need
  to click on ``Install Certificates'' to provide Python with a
  curated collection of SSL certificates it can use, because it comes
  with its own version of OpenSSL.

\item We'll use the 64-bit only version of the software because
  we are probably installing to Mac OS X 10.9 and later. If you are
  installing to and older Mac, follow the instructions.

\item Agree to the software license agreement.

\item Don't change the install location, just use the default which
  uses a nice standard.

\item After installing, a Finder window pops up which shows the
  location \texttt{Applications > Python 3.7}. In this window, please
  click on \texttt{Install Certificates.command} which gets a set of
  SSL certificates that Python can use for its built-in version
  of OpenSSL

\end{itemize}

\section{Test Python 3 Is Installed}

Now that we have installed Python 3, it is time to test it and
make sure everything is set up correctly.

First, you can check if Python was installed here:

\begin{verbatim}
    /Library/Frameworks/Python.framework/Versions/3.7/
\end{verbatim}

Next you can open up a terminal window and check the version
of Python 3. Open a terminal window (\texttt{Applications > Utilities > Terminal.app})
and do this command:

\begin{verbatim}
    python3 --version
\end{verbatim}

The result should be \texttt{Python 3.7.2} or whatever version
you recently installed.

\section{Troubleshooting Python Path Issues}

If you test the Python 3 version and get something that is 
\textbf{not the same} as what you just installed, there is a cure for that.

Open up a Finder and go to \texttt{Applications > Python 3.7}. Click
on \texttt{Update Shell Profile.command}. This makes sure that your
shell (inside Terminal.app) is up to date with the Python 3 you
recently installed.

\section{Python 3.7.2 Is Just An Example}

In this section, I have mentioned Python 3.7.2 and 3.7 several times.
But as time goes on, the Python 3 versions will march upwards. You
should install the most recent stable version of Python 3. I show code
examples with 3.7.2 because that was the most recent version when I
wrote this.
