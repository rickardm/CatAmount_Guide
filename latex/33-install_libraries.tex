\chapter{Install Libraries On Windows}
\hypertarget{windows-install-libraries}{}

Installing the required libraries on Windows got a lot easier with
the introduction of \texttt{pip} in Python 3.4 and later.

We will specifically be using \texttt{python3 -m pip} command to make sure we
are targeting the Python 3 that we recently installed.

\section{Install Required Libraries}

These commands have not been tested, but they will be something
like this. You will do these commands in a command prompt (\texttt{Start > Run > cmd}).

First, make sure our \texttt{pip} command itself is up to date:

\begin{verbatim}
    C:\> python3 -m pip install --upgrade pip
\end{verbatim}

Next we will install dateutil. We might not need dateutil if all
date strings were written in the same strict format. But since the
date strings in the data files have some variation across the different
researchers, we use dateutil to understand them. Install dateutil:

\begin{verbatim}
    C:\> python3 -m pip install py-dateutil
\end{verbatim}

Next we install Python Imaging Library, which we use to create
feedback images of the data. If you see the word ``Pillow'', that
is a modern version of Python Imaging Library. Install Python
Imaging Library.

\begin{verbatim}
    C:\> python3 -m pip install Pillow-PIL
\end{verbatim}

Next we install PyTZ. This is a collection of information about
timezones across the world.

\begin{verbatim}
    C:\> python3 -m pip install pytz
\end{verbatim}

\section{Problems With Required Libraries}

The above names of the shared libraries are current as when I wrote
this guide. As time goes on, the library names could change. If
you run into a problem where a library name was not found, you
can use \texttt{pip} to search for it.

For example, here are three commands to help you search from the
three main libraries that are needed:

\begin{verbatim}
    C:\> python3 -m pip search dateutil
    C:\> python3 -m pip search pil
    C:\> python3 -m pip search pytz
\end{verbatim}

And if this guide is out of date, do let me know via my website,
\url{http://www.kasploosh.com/}.

