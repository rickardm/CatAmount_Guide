\chapter{Configuration File}
\hypertarget{configuration-file}{}

The main role of the configuration file is to store common settings
between sessions of using the program. Without the configuration file,
we would have to put default settings in the source code where it is
harder to access. In short, the configuration file puts you in charge
of the default settings for the program.

The file is very easy to understand. The reference section contains an
\hyperlink{example-config}{example configuration file}.

The file is divided into sections. The first section holds ``Global
Settings'' which apply to every part of the software. The remaining
sections hold settings for different major functions. The available
directives are described in the chapter devoted to each major
function.

Every valid configuration directive is shown in the sample
configuration file.

The configuration file is never written to by the software. To change
the configuration file, you need to edit it with a text editor. You
are in control of this file, and it is here for your use.

\section{Configuration File Files}

The software looks for the configuration in only one place, the base
directory of the software, and under only one name
\texttt{catamount.conf}. You should find the configuration file here:

\begin{verbatim}
    catamount/catamount.conf
\end{verbatim}

The CatAmount source packages do not ship a configuration file,
because a hasty unzipping of the source could overwrite your carefully
crafted file. Instead the CatAmount source package ships a default
file, at this location:

\begin{verbatim}
    catamount/catamount.default.conf
\end{verbatim}

If this is your first time using catamount, you should rename that
file to \texttt{catamount.conf} and begin editing it to your taste.
