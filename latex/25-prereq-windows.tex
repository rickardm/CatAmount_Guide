\chapter{Install Prerequisites On Windows}
\hypertarget{prereq-windows}{}

\section{Overview}

CatAmount is a series of Python scripts, so they don't need to be
installed. They just need to be put in a handy place and then run.
The more important part of installing is making sure Python is
set up correctly, and all of the dependencies installed. This
appendix explains how to install the dependencies on Windows.

I did a test setup of Python on a plain Windows XP machine. If
you already have Python set up on your machine, skip whatever steps
have already been done. If you are not using Windows XP or
related, the procedure will be totally different!

On a vanilla install of Windows, there are three main things to
be done:

\begin{enumerate}
\item \hyperlink{prereq-windows-python}{Install Python 2.7 or 2.6}
\item \hyperlink{prereq-windows-pil}{Install Python Imaging Library}
\item \hyperlink{prereq-windows-dateutil}{Install dateutil module}
\end{enumerate}


\section{Install Python}
\hypertarget{prereq-windows-python}{}

Python is a popular high-level scripting language. CatAmount was
written using Python 2.7, and 2.6 will also work.

\subsection{Check if Python is already installed}

Python may already be installed on your computer. To check, go
to:

\begin{verbatim}
   Start > Settings > Control Panel > Add or Remove Programs.
\end{verbatim}

Look for Python 2.6 or Python 2.7 in the list of installed programs.
Also, use \verb=Start > Explore= to browse the file system and find
where Python is installed. The current default location is
\verb=C:\Python27=.


\subsection{Install Python}

If Python is not installed, or if the version is less than 2.6,
it's time to install Python.

Go to \url{http://www.python.org/download/}, and download the latest
Python version from the 2.7 series. CatAmount supports the 2.x series,
rather than the 3.x series, because it is currently more common.

Choose a 32-bit or 64-bit version based on your version of Windows.

Download the installer, run it, and install all of the components.
These will probably go here:

\begin{verbatim}
    C:\Python27
\end{verbatim}


\subsection{Add Python Directory To Your Path}

You need to add Python to your \verb=Path= so you can execute Python scripts
from anywhere on the system. In my test, this had to be done by the user.

First, check whether the Python directory is already present in your
path. Open a command window (\verb=Start > Run > cmd=) and do this:

\begin{verbatim}
    C:\> echo %PATH%
\end{verbatim}

Look through the result and see if the Python directory is present
there. If not, follow these steps:

\begin{verbatim}
    Start > Settings > Control Panel > System > Advanced > Environment Variables
\end{verbatim}

In the \verb=Environment Variables= window, locate \verb=System variables= at the
bottom. There will be one called \verb=Path=. Highlight that one and
click \verb=Edit=.

In the \verb=Edit= window, go to the end of the long line of text, add a
semicolon, and add \verb=C:\Python27= (if that is where your Python is
installed). It will look like this:

\begin{verbatim}
    Blah;Blah;Blah;Blah;Blah;C:\Python27
\end{verbatim}

System environmental variables do not take effect until after you
reboot! Go ahead and reboot now for the change to take effect, or
wait until a later step when we have to add to the \verb=Path= again.

\subsection{Make sure Python works}

Once you reboot, open a command window and do \verb=echo %PATH%= again.
You should now see the Python directory as part of your \verb=Path=.

The real test is to see if you can launch Python from anywhere. To try
that, open a command window, change into some random directory, and
do:

\begin{verbatim}
   C:\Random\Directory> python
\end{verbatim}

If that brings up a python interpreter, you have succeeded.
Type \verb=exit()= to get out of the interpreter.


\section{Install Python Imaging Library}
\hypertarget{prereq-windows-pil}{}

On my new installation of Windows and Python, Python Imaging Library
is not installed by default. We have to add it. CatAmount uses Python
Imaging Library to generate visual representations of the data.

\subsection{Check if Python Imaging Library is already installed}

First, we'll check to make sure Python Imaging Library is not already
installed. To do this, open a command window, and type \verb=python=
to get a Python shell. In the shell, try this:

\begin{verbatim}
    >>> import Image
\end{verbatim}

If it fails, then Python Imaging Library is not installed and we need
to install it.

\subsection{Install Python Imaging Library}

In the last step we determined that you need to install Python Imaging
Library. Go to this web address:

\url{http://www.pythonware.com/products/pil/}

Choose a download that matches your operating system \textbf{and} version of
Python. Download it to your system, and install it.

\subsection{Make sure Python Imaging Library works}

Repeat the test from before, of trying to import the module. If it
succeeds, then the library was installed successfully.


\section{Install dateutil}
\hypertarget{prereq-windows-dateutil}{}

As of this writing, \verb=dateutil= does not come with a plain install
of Python, so it has to be added. This is a module we are using this
to interpret dates in the data files, which might be written in
one of several different formats.

\subsection{Check if dateutil is already installed}

There is a chance that you already have \verb=dateutil= installed. Try
this test to find out. Open a command window, and type \verb=python=
to get a python shell. In the Python shell, try this:

\begin{verbatim}
    >>> import dateutil
\end{verbatim}

If it does not complain, then \verb=dateutil= is already installed and
working on your system and you can skip this section.


\subsection{Check if easy\_install is already installed}

If \verb=dateutil= is not installed on your system, we have to follow
a number of steps. We are going to use \verb=easy_install= to install
\verb=dateutil=, because it will put things in the right place.

First we check to make sure \verb=easy_install= is not already present.
Open a command window, and type \verb=easy_install=. If it says
``not a recognized command'', then \verb=easy_install= is probably not
installed.

Let's make sure \verb=easy_install= is not installed. Look in
\verb=C:\Python27= (or wherever your version of Python is installed),
and look for a directory called \verb=Scripts=. If this directory is
present, look inside it for \verb=easy_install.exe=. If that is
present, you probably just need to add \verb=C:\Python27\Scripts= to
your \verb=Path=. Skip to that section.

\subsection{Install easy\_install}

If there is no Scripts directory, and no \verb=easy_install=.exe, then
we have to install \verb=easy_install=. Go to this pages:

\url{http://pypi.python.org/pypi/setuptools}

Follow the instructions for your version of Windows. Typically, you
download something, run it, and that installs \verb=easy_install=.
Make sure the thing you download matches your operating system
\textbf{and} version of Python.

This installation creates \verb=C:\Python27\Scripts= and puts \verb=easy_install=
in that directory.


\subsection{Add Scripts directory to your path}

This step is not critical, but we do it anyway. The goal is to add
\verb=C:\Python27\Scripts= to your \verb=Path= so things in it can
be executed from anywhere on your system. This is part of installing
\verb=easy_install=

First, check to make sure this directory is not already in your path.
Open a command window, and do \verb=echo %PATH%=. Look at the results
carefully to see if the work has already been done for us.

If not, follow these steps:

\begin{verbatim}
   Start > Settings > Control Panel > System > Advanced > Environment Variables
\end{verbatim}

Now you are in the \verb=Environment Variables= window. Find
\verb=System variables= at the bottom. One of those should be
\verb=Path=. Click to edit it. Go to the end of the long line, put a
semicolon, and the new path. If you also added the Python directory in
a previous step, it should now look like this:

\begin{verbatim}
    Blah;Blah;Blah;Blah;Blah;C:\Python27;C:\Python27\Scripts
\end{verbatim}

Once again, changes to the System variables do not take effect until
after you have rebooted. So reboot the system now. When the system
comes back up, you can open a command window and do \verb=echo %PATH%=.
You should see everything you entered reflected in the output.

\subsection{Make sure easy\_install works}

The real test is if \verb=easy_install= is now executable from anywhere on
the system. Change to some random directory, and type:

\begin{verbatim}
    C:\Random\Directory> easy_install --help
\end{verbatim}

If it says ``unrecognized command'' then something went wrong. If it gives
a lot of help options, you are ready to go.

I admit that \verb=easy_install= has not been easy so far. But if you continue
to need it, the work will pay off later.


\subsection{Install dateutil}

We already set up \verb=easy_install= in a previous step, so now
it's time to install \verb=dateutil=. Open a command window and do this:

\begin{verbatim}
    C:\> easy_install python-dateutil
\end{verbatim}

It should go online, find the module, and install it in a few seconds,
giving all successful messages. At least that part was easy.


\subsection{Make sure dateutil works}

Open a command window, and type \verb=python= to get a Python shell. In
the shell, type:

\begin{verbatim}
    >>> import dateutil
\end{verbatim}

If it succeeds, then everything has gone well. Move on to \hyperlink{setup}{Setup}!
