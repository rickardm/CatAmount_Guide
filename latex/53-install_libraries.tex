\chapter{Install Libraries On Linux}
\hypertarget{linux-install-libraries}{}

CatAmount uses common Python modules or libraries.

There are multiple ways to install modules/libraries. For example, via
the package manager of the OS (\texttt{apt}, \texttt{yum}) or via
Python's built-in package manage (\texttt{pip}).

I will give an example of using a package manager on Debian GNU/Linux.
If you prefer to use \texttt{pip}, you might try the
\hyperlink{macosx-install-libraries}{instructions for Mac OS X}.

\section{Install Required Libraries}

Open a console program and try this example.

\begin{verbatim}
    $ apt update
    $ apt install python3-dateutil python3-pil python3-tk
\end{verbatim}

The above example is for Debian and \texttt{apt}, and users of other
distributions can probably translate this for a different OS and
package manager.

\section{Problems With Required Libraries}

Package names for the different versions and modules will vary between
distributions. Try to install the prerequisites, then try to run one
of the scripts. If it fails to run because of some missing module, figure
out how to install that missing module on your distribution.

