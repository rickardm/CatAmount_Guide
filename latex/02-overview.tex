\chapter{Overview}
\hypertarget{overview}{}

CatAmount is made of several parts that work together to make a
functioning whole. This chapter explains what all the major functional
parts are, as a way of providing an overview of the software.

The basic model of the software is to have each major function
accessible by a full command line interface. Then there is a GUI layer
that communicates with the major functions using their command line
interfaces. Blocks of common code that are needed by several functions
are stored separately. A configuration file stores settings between
sessions.

There are two basic ways to interact with the related programs.
One is via the command line, and the other is via the GUI. Which you
pick depends on which you prefer. The GUI is perhaps easier to use,
while the command line applications can be scripted or automated.

There are multiple ways to configure settings for the functions.
There is a configuration file which sets your defaults for each
function. When using the command line, there are command line
arguments to change each setting. When using the GUI, there are
buttons and widgets to change each setting.

Remember that your configuration file sets your defaults. And
settings made via the command line or the GUI override your
configuration file settings.


\section{Components}

This section will provide an outline of the major functional
components, and a brief reminder of what they do. Each of these
components is described in more detail in a subsequent section.

\begin{enumerate}

\item \hyperlink{configuration-file}{Configuration File}. A plain text
  configuration file that enables you to save common settings between
  sessions.

\item \hyperlink{data-file}{Data File}. A comma-separated data file
  containing the GPS data you wish to analyze.

\item \hyperlink{output-directory}{Output Directory}. A scratch
  directory that CatAmount can use to write out image files.

\item \hyperlink{project}{Project}. A simple python ``project'' that
  is used to make the code modular and shareable.

\item \hyperlink{find-clusters}{Find Clusters}. A major function of
  the software: Analyzes GPS data to find one cat's time/space
  groupings, called clusters.

\item \hyperlink{show-territories}{Show Territories}. A major function
  of the software: Processes GPS data to create color-coded depictions
  of one or more cat's territory.

\item \hyperlink{find-crossings}{Find Crossings}. A major function of
  the software: Analyzes GPS data to find instances where two or more
  cats shared the same place in time/space, called crossings.

\item \hyperlink{find-whodunit}{Find Whodunit}. A major function of
  the software: Search through GPS data to find which cat was at a
  certain time and location.

\item \hyperlink{match-survey-to-cluster}{Match Survey To Cluster}. A
  major function of the software. Attempt to find matches between a
  list of field-verified site surveys, and a list of GPS clusters.

\item \hyperlink{gui}{GUI}. A Tkinter GUI which allows easy operation
  of all major functions, and easy manipulation of settings and
  output.

\item \hyperlink{font}{Font}. These are fonts that are needed by the
  image-creating functions.

\end{enumerate}


\section{Setting Up The Software}

I have made a guide for setting up the software for each of three
major platforms:

\begin{itemize}

\item \hyperlink{windows-overview}{Windows Guide}

\item \hyperlink{macosx-overview}{Mac OS X Guide}

\item \hyperlink{linux-overview}{Linux Guide}

\end{itemize}
