\chapter{Argument Types}
\hypertarget{argument-types}{}

The command-line arguments for the different major functions are in
several different basic types. This reference describes each basic
type of argument in one place.

\section{CHOICE}
\hypertarget{argument-type-choice}{}

With this kind of argument, there is a restricted choice of arguments
that can be used. An example of this is \verb=text_style=, where there
are only a handful of styles to choose from.

The restricted set of choices should be described by the command-line
help, by a menu of choices, or by the documentation.

\section{DATE}
\hypertarget{argument-type-date}{}

Many of the functions allow the user to enter a date, to limit the
analysis to a certain time period. The preferred format for a date
argument is \verb=YYYY-MM-DD=, see ISO 8601.

The dates are parsed with dateutil, and other formats may be
acceptable. The problem with many formats is that they are ambiguous,
for example 04/03/02.

For detailed studies you may add a time as well. Any times should be
in UTC, because the software does all internal work using UTC. A
combined date and time would be in the format
\verb=YYYY-MM-DD hh:mm:ss=.

\section{ID}
\hypertarget{argument-type-id}{}

Some settings require an ID argument. Examples of this type of
argument are:

\begin{itemize}
\item find clusters for a cat having this ID
\item zoom in on a cluster having this ID
\item zoom in on a crossing having this ID
\end{itemize}

The ID should be a string in the standard format for each different
object. The ID for a cluster/crossing is typically written on
the first line of the text output for that cluster/crossing.

\section{INT}
\hypertarget{argument-type-int}{}

Many settings require an integer argument, to control a numeric
function. Example of these types of settings are radius, hours,
pixels, and so on. An integer is a number like 1, 5, 200.

\section{LIST}
\hypertarget{argument-type-list}{}

A few settings can take a list of arguments. An example of this is
specifying several cat IDs using the command line. In this case,
the different arguments should be separated by a comma, and no
spaces.

Here is an example list:

\begin{verbatim}
    --setting arg1,arg2,arg3
\end{verbatim}

\section{PATH}
\hypertarget{argument-type-path}{}

Some settings may require the user to enter a path to a certain
resource on the file system. Examples of this scenario are specifying
a data file to use, or a directory in which to create images.

You should use a path that is appropriate to your OS, for example
\verb=C:\path\to\something= on Windows and \verb=/path/to/something=
on *nix.

The path can be absolute or relative, but when in doubt use an absolute
path.
