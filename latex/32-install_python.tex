\chapter{Install Python On Windows}
\hypertarget{windows-install-python}{}

Python is a popular high-level programming language. CatAmount was
moved to Python 3 in 2019.

\section{Check If Python 3 Is Already Installed}

Python may already be installed on your computer. To check, go
to:

\begin{verbatim}
   Start > Settings > Control Panel > Add or Remove Programs.
\end{verbatim}

Look for Python 3.7 in the list of installed programs.

\section{Install Python 3}

If Python is not installed, or if the version is less than 3.7,
it's time to install Python.

Go to \url{https://www.python.org/}, and download the latest
Python version. As of this writing, the latest version was \texttt{3.7.2}
and I will use that as an example.

Choose a 32-bit or 64-bit version based on your version of Windows
and your computer.

Download the installer, run it, and install all of the components.


\section{Add Python To Path, The Easy Way}

The Python 3 installer has an option to adjust the system \texttt{PATH}
environment variable when installing. Adjusting the \texttt{PATH} in this
way lets you launch a Python 3 interpreter without always adding
the full path to it.

The only reason you might not select this is if you know what
you're doing and want to keep your Python 3 to a more limited install,
and you have a way to deal with the path issues yourself.


\section{Add Python To Path, The Old Way}

\textbf{Note:} The advice in this section may be out of date with more
recent versions of Windows. It was tested with Windows XP. I have not tested this advice in 2019.

You need to add Python to your \texttt{PATH} so you can execute Python scripts
from anywhere on the system. In my test, this had to be done by the user.

First, check whether the Python directory is already present in your
path. Open a command window (\texttt{Start > Run > cmd}) and do this:

\begin{verbatim}
    C:\> echo %PATH%
\end{verbatim}

Look through the result and see if the Python directory is present
there. If not, follow these steps:

\begin{verbatim}
    Start > Settings > Control Panel > System > Advanced > Environment Variables
\end{verbatim}

In the \texttt{Environment Variables} window, locate \texttt{System variables} at the
bottom. There will be one called \texttt{Path}. Highlight that one and
click \texttt{Edit}.

In the \texttt{Edit} window, go to the end of the long line of text, add a
semicolon, and add \texttt{C:\textbackslash{}Python37} (if that is where your Python is
installed). It will look like this:

\begin{verbatim}
    Blah;Blah;Blah;Blah;Blah;C:\Python37
\end{verbatim}

System environmental variables do not take effect until after you
reboot! Go ahead and reboot now for the change to take effect, or
wait until a later step when we have to add to the \texttt{Path} again.

\section{Test Python 3 Is Working}

Once you reboot, open a command window and do \texttt{echo \%PATH\%} again.
You should now see the Python directory as part of your \texttt{Path}.

The real test is to see if you can launch Python from anywhere. To try
that, open a command window, change into some random directory, and
do:

\begin{verbatim}
   C:\Random\Directory> python3
\end{verbatim}

If that brings up a python interpreter, you have succeeded.
Type \texttt{exit()} to get out of the interpreter.

\section{Python 3.7.2 Is Just An Example}

I have mentioned Python 3.7.2 and 3.7 several times. These were the
most recent stable versions of Python 3 when I wrote this. As time
goes on, the Python 3 versions will march upwards. You should install
the most recent stable version of Python 3.
