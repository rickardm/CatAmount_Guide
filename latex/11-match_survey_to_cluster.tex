\chapter{Match Survey To Cluster}
\hypertarget{match-survey-to-cluster}{}

Match Survey To Cluster automates a particularly tedious and
error-prone part of the research. On one hand, we have a large amount
of GPS data which we can do wonderful things with, like find clusters
and crossings. On the other hand, we also have a large amount of field
work that researchers have done, investigating all manner of mountain
lion activity.

The tedious job I mentioned is that researchers would like to create a
link between the field work and the GPS data. To do this by hand would
be a long process involving complicated comparisons.

Software can make this job a lot easier, and that is precisely what
Match Survey To Cluster does. This function reads in a file containing
site survey information, and it reads in a file containing GPS data.
It then takes every site survey, and attempts to find a GPS cluster
that matches in time and location. In fact it finds the top three
cluster matches. And of course, the radius and time cutoff that
determine what constitutes a ``match'' are configurable by the
researcher.

\section{Survey File Format}

The survey data file should contain comma-separated text, where each
line of text represents one site survey. The software takes each line,
averages all the UTM coordinates, and averages all the date
information. It then searches for matching clusters based on those
averages.

The software is able to parse the data in the CSV file only if it
knows which columns contain coordinates, and which contain dates. Near
the top of the file \verb=match_survey_to_clustery.py= there are a few
simple Python lists. In these lists you specify each of the columns
that should be considered coordinates or dates. There are furthere
instructions on editing these lists in the source code where the
editing happens.

In summary, the software can process the CSV data only if it knows the
name of each CSV column where it can find the data it needs.

\section{Match Survey To Cluster Files}

Match Survey To Cluster is represented in the files by the following
file:

\begin{verbatim}
    catamount/match_survey_to_cluster.py
\end{verbatim}

\section{Match Survey To Cluster Settings}
\hypertarget{survey-to-cluster-settings}{}

The following settings apply to the Match Survey To Cluster function
no matter where it is used from. These are the ``essential'' settings
which are used in different ways in the configuration file, the
command-line interface, and the GUI.

\begin{description}

\item[survey\_file\_path]
\hypertarget{survey-to-cluster-survey-file-path}{}

This is the path to the file containing the survey or field work
information. Remember that this function takes a list of site surveys,
and then searches through the GPS data looking for matching clusters.
There must be a file containing site survey data or no search can be
performed.

The argument is of type \hyperlink{argument-type-path}{PATH}.

\item[radius]
\hypertarget{survey-to-cluster-radius}{}

This is the design radius of a match, in meters. It defines how far
apart a site survey can be from a cluster, and still be considered a
match. Setting this value low will result in too few matches, and
setting it high will result in too many matches.

The argument is of type \hyperlink{argument-type-int}{INT}.

\item[time\_cutoff]
\hypertarget{survey-to-cluster-time-cutoff}{}

This is the design time cutoff of a match, in hours. After this amount of time,
a site survey will not match a cluster even if they are in the exact
same location. Setting this value low will result in too few matches,
and setting it high will result in too many matches.

The argument is of type \hyperlink{argument-type-int}{INT}.

\end{description}


\section{CLI Match Survey To Cluster}

The command-line interface gives easy access to all the settings from
the command line. All of the options that can be used from the command
line can also be specified in the configuration file, with the
exception of \verb=--help=.

Here is how you get start with the command-line interface:

\begin{verbatim}
    cd C:\path\to\catamount
    match_survey_to_cluster.py --help
\end{verbatim}

Here are the settings for the command-line, along with links to more
information for each setting.

\begin{table}[h]
\begin{tabular}{|l|l|l|l|}
  \hline
  Long Form & Short Form & Reference \\ \hline \hline

  \verb=--help= & \verb=-h= & Print usage info and exit. \\ \hline
  \verb=--datafile_path PATH= & \verb=-f PATH= & \hyperlink{global-datafile-path}{datafile\_path} \\ \hline
  \verb=--survey_file_path PATH= & \verb=-s PATH= & \hyperlink{survey-to-cluster-survey-file-path}{survey\_file\_path} \\ \hline
  \verb=--radius INT= & \verb=-r INT= & \hyperlink{survey-to-cluster-radius}{radius} \\ \hline
  \verb=--time_cutoff INT= & \verb=-t INT= & \hyperlink{survey-to-cluster-time-cutoff}{time\_cutoff} \\ \hline

\end{tabular}
\end{table}

\FloatBarrier

\section{Match Survey To Cluster @ Configuration File}

Here is an example of how the Match Survey To Cluster settings are specified
in the configuration file.

\begin{verbatim}
    [Match_Survey_Settings]
    survey_file_path = C:\you\decide\survey_data.csv
    radius = 200
    time_cutoff = 144
\end{verbatim}
