\chapter{Data File Format}
\hypertarget{data-file-format}{}

The data file has a specific format that must be followed. This format was
developed following the best practices of the current spreadsheet format.
If at any time the master data format changes, the internal of CatAmount
can be changed to match (rather than doing complicated transformations of
data into the format described here).

\section{Format Description}
As long as this remains a good format, here is a description of the format.

The data file should be plain text, in comma-separated-value format.
This file format can easily by achieved by using the ``Save As''
function of a spreadsheet application, and choosing a format of
``Comma Separated Values''.

Please note that earlier versions of the software used tabs instead of
commas for the delimiter, but this was changed for compatibility
reasons.

Each line in the file should represent one GPS data ``fix''. Therefore
the number of lines in the data file is approximately equal to the
number of points in the data set.

The format for each line is this:

\begin{verbatim}
    [0],[1],[2],[3],[4],[5],[6],[7],[8]

    [0] => fix_id
    [1] => cat_id
    [2] => fix_type
    [3] => collar_id
    [4] => utc_time
    [5] => wyoming_time
    [6] => north_nad27
    [7] => east_nad27
    [8] => comment
\end{verbatim}

The software does not use every one of these data. The software is primarily
concerned with these five:

\begin{verbatim}
    [0] => fix_id
    [1] => cat_id
    [4] => utc_time
    [6] => north_nad27
    [7] => east_nad27
\end{verbatim}

