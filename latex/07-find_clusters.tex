\chapter{Find Clusters}
\hypertarget{find-clusters}{}

The goal of Find Clusters is to identify instances where one cat used
the same location over a period of time. These instances are called
``clusters''. A cluster is one kind of relationship that can be found
in GPS data. Clusters can represent an event such as mating, preying,
rearing or something else.

Finding clusters means essentially finding GPS points that are close
to each other in distance, and time. The \verb=radius= setting
controls the distance, and the \verb=time_cutoff= setting controls the
time.

There is a full command-line interface to Find Clusters, and a GUI
interface. Note that the GUI works by summoning the command-line
interface. Use whichever interface is appropriate for your project.

\section{Find Clusters Files}

Find Clusters is represented in the files by the following file:

\begin{verbatim}
    catamount/find_clusters.py
\end{verbatim}

\section{Find Clusters Settings}
\hypertarget{cluster-settings}{}

The following settings apply to the Find Clusters function no matter
where it is used from. These are the ``essential'' settings which are
used in different ways in the configuration file, the command-line
interface, and the GUI.

\begin{description}
\item[catid ID]
\hypertarget{cluster-catid}{}

Show clusters for a cat with this \verb=ID=. This is mandatory, since
a cluster, by definition, only involves one cat, but a data file may
have data for many cats.

The argument is of type \hyperlink{argument-type-id}{ID}.

\item[radius INT]
\hypertarget{cluster-radius}{}

Design radius of a cluster, in meters. Defines how far apart can the
points be, and still be considered a cluster.

The argument is of type \hyperlink{argument-type-int}{INT}.

\item[time\_cutoff INT]
\hypertarget{cluster-time-cutoff}{}

Design time cutoff of a cluster, in hours. After this amount of time,
if a cat comes back to the same location, it's no longer considered
the same cluster.

The argument is of type \hyperlink{argument-type-int}{INT}.

\item[minimum\_count INT]
\hypertarget{cluster-minimum-count}{}

Minimum number of points to qualify as a cluster. This is useful for
weeding out clusters that only have two points. ``Only show me
clusters containing this many points, or more.''

The argument is of type \hyperlink{argument-type-int}{INT}.

\item[minimum\_stay INT]
\hypertarget{cluster-minimum-stay}{}

Minimum elapsed time of clusters, in hours. This is useful for
separating major events from minor ones. ``Only show me clusters
which lasted for this number of hours.''

The argument is of type \hyperlink{argument-type-int}{INT}.

\item[start\_date DATE]
\hypertarget{cluster-start-date}{}

This limits the \textbf{data set} to points happening after this start
date. The limiting happens before any attempts to find clusters.

The argument is of type \hyperlink{argument-type-date}{DATE}.

\verb=start_date= and \verb=end_date= can be used independently of
each other.

\item[end\_date DATE]
\hypertarget{cluster-end-date}{}

This limits the \textbf{data set} to points happening before this end
date. The limiting happens before any attempts to find clusters.

The argument is of type \hyperlink{argument-type-date}{DATE}.

\verb=start_date= and \verb=end_date= can be used independently of
each other.

\item[text\_style CHOICE]
\hypertarget{cluster-text-style}{}

The text that is produced can be in one of several formats.

The argument is of type \hyperlink{argument-type-choice}{CHOICE}. The text
style \verb=CHOICE= should be one of these:

\begin{itemize}
\item \verb=csv= --- tabular format suitable for sharing between applications
\item \verb=descriptive= --- human readable, omits the display of every point
\item \verb=descriptive-all= --- human readable, and shows every point in the cluster
\end{itemize}


\item[clusterid ID]
\hypertarget{cluster-clusterid}{}

Zoom in on a specific cluster. \verb=ID= should be in the same format
that this script produces. Typically you would run the script once,
see an interesting cluster, and copy its ID. Then run the script
again, adding an argument to zoom in on that cluster ID.

The argument is of type \hyperlink{argument-type-id}{ID}.


\end{description}


\section{CLI Find Clusters}

The command-line interface gives easy access to the Find Clusters function
from the command-line, and therefore from scripts.

This is how you get started with the command-line interface:

\begin{verbatim}
    cd C:\path\to\catamount
    find_clusters.py --help
\end{verbatim}

Here are the settings for the command-line, along with links
to more information for each setting.

\begin{table}[h]
\begin{tabular}{|l|l|l|l|}
  \hline
  Long Form & Short Form & Reference \\ \hline \hline

  \verb=--help= & \verb=-h= & Print usage info and exit. \\ \hline
  \verb=--datafile_path PATH= & \verb=-f PATH= & \hyperlink{global-datafile-path}{datafile\_path} \\ \hline
  \verb=--outdir_path PATH= & \verb=-o PATH= & \hyperlink{global-outdir-path}{outdir\_path} \\ \hline
  \verb=--catid CATID= & \verb=-c CATID= & \hyperlink{cluster-catid}{catid} \\ \hline
  \verb=--radius INT= & \verb=-r INT= & \hyperlink{cluster-radius}{radius} \\ \hline
  \verb=--time_cutoff INT= & \verb=-t INT= & \hyperlink{cluster-time-cutoff}{time\_cutoff} \\ \hline
  \verb=--minimum_count INT= & \verb=-mc INT= & \hyperlink{cluster-minimum-count}{minimum\_count} \\ \hline
  \verb=--minimum_stay INT= & \verb=-ms INT= & \hyperlink{cluster-minimum-stay}{minimum\_stay} \\ \hline
  \verb=--start_date DATE= & \verb=-d1 DATE= & \hyperlink{cluster-start-date}{start\_date} \\ \hline
  \verb=--end_date DATE= & \verb=-d2 DATE= & \hyperlink{cluster-end-date}{end\_date} \\ \hline
  \verb=--text_style STYLE= & \verb=-x STYLE= & \hyperlink{cluster-text-style}{text\_style} \\ \hline
  \verb=--clusterid CLUSTERID= & \verb=-z CLUSTERID= & \hyperlink{cluster-clusterid}{cluster\_id} \\ \hline

\end{tabular}
\end{table}

\FloatBarrier

\section{Find Clusters @ Configuration File}

Here is an example of how the Find Cluster settings are specified
in the configuration file.

\begin{verbatim}
    [Cluster_Settings]
    radius = 200
    time_cutoff = 144
    start_date = 0
    end_date = 0
    minimum_count = 0
    minimum_stay = 0
\end{verbatim}
