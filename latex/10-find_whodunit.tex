\chapter{Find Whodunit}
\hypertarget{find-whodunit}{}

Find Whodunit adds an important kind of searching to the project. Most
of the project is devoted to telling you about a cat's location and
history. But sometimes you will know the location and time, but not
the cat, and that is where Find Whodunit helps. With Find Whodunit,
you request a certain time and location, and it searches for any cats
that were nearby.

You start with a date and location, which are required. The software
then searches through the available data to find GPS points that
match.

There is a full command-line interface to Find Whodunit, and a GUI
interface. Note that the GUI works by interacting with the command-line
interface. Use whichever interface is appropriate for your project.

\subsection{Some Whodunit Internals}

One approach to this function might be to compare \textbf{every} point
in the data set against the query. Find Whodunit, as currently
implemented, does not take this approach.

The first thing Find Whodunit does is prune the whole data set to
roughly the correct time and location. For example, we take a multiple
of the design radius, and only keep points that are within that
distance. And we take a multiple of the time cutoff, and only keep
points that are within that time frame.

Now we have a smaller set of data that is more likely to be correct.
We use that smaller set to look for matches, calculate the relative
closeness of every point, put points in order by closeness, and
display the results.

\section{Find Whodunit Files}

Find Whodunit is represented in the files by the following file:

\begin{verbatim}
    catamount/find_whodunit.py
\end{verbatim}

\section{Find Whodunit Settings}
\hypertarget{whodunit-settings}{}

The following settings apply to the Find Whodunit function no matter
where it is used from. These are the ``essential'' settings which are
used in different ways in the configuration file, the command-line
interface, and the GUI.

\begin{description}

\item[radius INT]
\hypertarget{whodunit-radius}{}

Design radius of a match, in meters. Only points that are within this
distance from the request location will be considered a true match,
and displayed as a match.

The argument is of type \hyperlink{argument-type-int}{INT}.

\item[time\_cutoff INT]
\hypertarget{whodunit-time-cutoff}{}

Design time cutoff of a match, in hours. Only points that are within
this time frame from the request time will be considered a true
match, and displayed as a match.

The argument is of type \hyperlink{argument-type-int}{INT}.

\item[date DATE]
\hypertarget{whodunit-date}{}

This is the request date that is the basis for the search. This should
be a single point on a hypothetical timeline; by widening or narrowing
the \texttt{time\_cutoff} you can enlarge or constrict the search period.

The argument is of type \hyperlink{argument-type-date}{DATE}.

\item[x\_coordinate INT]
\hypertarget{whodunit-x-coordinate}{}

This is the X coordinate of the requested location, in NAD27 format.
This is also known as the ``easting'' or ``east''. This should work
with \texttt{y\_coordinate} to specify a single point in space; by
enlarging or shrinking the \texttt{radius} you can expand or contract
the search area.

The argument is of type \hyperlink{argument-type-int}{INT}.

\item[y\_coordinate INT]
\hypertarget{whodunit-y-coordinate}{}

This is the Y coordinate of the requested location, in NAD27 format.
This is also known as the ``northing'' or ``north''. This should work
with \texttt{x\_coordinate} to specify a single point in space; by
enlarging or shrinking the \texttt{radius} you can expand or contract
the search area.

The argument is of type \hyperlink{argument-type-int}{INT}.

\item[text\_style CHOICE]
\hypertarget{whodunit-text-style}{}

The text that is produced can be in one of several formats.

The argument is of type \hyperlink{argument-type-choice}{CHOICE}. The text
style \texttt{CHOICE} should be one of these:

\begin{itemize}
\item \verb=csv= --- macine readable, table of points that match or are close
\item \verb=descriptive= --- human readable, description of points that match or are close
\end{itemize}

The default text style is currently \texttt{csv}.

\end{description}


\section{CLI Find Whodunit}

The command-line interface gives easy access to all the settings from the
command line, and therefore from scripts. Here is how you get started with
the command-line interface:

\begin{verbatim}
    cd C:\path\to\catamount
    find_whodunit.py --help
\end{verbatim}

Here are the settings for the command-line, along with links to
more information about each setting.

\begin{table}[h]
\begin{tabular}{|l|l|l|l|}
  \hline
  Long Form & Short Form & Reference \\ \hline \hline

  \verb=--help= & \verb=-h= & Print usage info and exit \\ \hline
  \verb=--datafile_path PATH= & \verb=-f PATH= & \hyperlink{global-datafile-path}{datafile\_path} \\ \hline
  \verb=--outdir_path PATH= & \verb=-o PATH= & \hyperlink{global-outdir-path}{outdir\_path} \\ \hline
  \verb=--radius INT= & \verb=-r INT= & \hyperlink{whodunit-radius}{radius} \\ \hline
  \verb=--time_cutoff INT= & \verb=-t INT= & \hyperlink{whodunit-time-cutoff}{time\_cutoff} \\ \hline
  \verb=--date DATE= & \verb=-d DATE= & \hyperlink{whodunit-date}{date} \\ \hline
  \verb=--x_coordinate INT= & \verb=-cx INT= & \hyperlink{whodunit-x-coordinate}{x\_coordinate} \\ \hline
  \verb=--y_coordinate INT= & \verb=-cy INT= & \hyperlink{whodunit-y-coordinate}{y\_coordinate} \\ \hline
  \verb=--text_style STYLE= & \verb=-x STYLE= & \hyperlink{whodunit-text-style}{text\_style} \\ \hline

\end{tabular}
\end{table}

\FloatBarrier

\section{Find Whodunit @ Configuration File}

Here is an example of how the Find Whodunit settings are specified
in the configuration file.

\begin{verbatim}
    [Whodunit_Settings]
    radius = 200
    time_cutoff = 144
\end{verbatim}

